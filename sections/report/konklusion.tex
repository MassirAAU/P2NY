\chapter{Konklusion}
\label{chap:konklusion}
Det kan hermed konkluderes, at konflikterne mellem cyklister og fodgængere i og omkring området Nytorv/Østerågade er gensidige. Som det konkluderes i afsnittet om resultater af observation, er hovedparten af alle konflikter, som finder sted ved fodgængerfeltet på Østerågade er alvorlige. Dette konkluderes sådan, eftersom cyklisterne og fodgængerne kommer tæt på hinanden ved nogle bestemte hastigheder. Motivationen til at undersøge konflikter mellem cyklister og fodgængere kom fra interviewsene, og skabte en yderligere nysgerighed. Der er blevet observeret både tidlige og sene samspil, men dog flest sene samspil, som teoretisk indikerer, at området omkring fodgængerfeltet er et usikkert område for fodgængerne. De typer konflikter, der primært finder sted, som det også tidligere i rapporten er blevet defineret, er mellem cyklister og fodgængere, som er på kollisionskurs.~\\
 Yderligere kan vi sige, at dette underbygges i form af vores observationer og trafiktællinger, da vi i vores trafiktælling af trafikantgrupperne fandt frem til, at fodgængerfeltet på Østerågade var det mest befærdet sted for alle cyklister, fodgængere og billister i hele området. Ud fra denne opdagelse valgte vi derfor at sætte vores fokus på netop denne location under vores observationer. Om hvorvidt området er Shared Space eller ej vurderer vi, at det er en blanding af både et Shared Space område og en sivegade, da der er elementer af begge begreber i området.~\\
 Et løsningsforslag, som eksempelvis det forslag der kommer i rapporten vedrørende en cykelbane, vurderer vi til at have en sandsynlig mulighed for at kunne løse eventuelle konflikter mellem cyklister og fodgængere, da cyklerne bliver på denne måde mere synliggjorte i trafikområdet. For at skabe et bedre trafik flow, mener vi, at det er nødvendigt at lave ændringer i området. En eventuel ændring, som ville kunne have en markant effekt, og dermed give et bedre trafik flow, kunne være at etablere en cykelbane i hele området. Dette ville som tidligere skrevet øge fokus omkring, hvor cyklerne er, da de bliver henvist til at køre inden for de afmærkede cykelbaner. Dette kan dog også have en negativ effekt, da vi er bange for, at det kan få cyklisterne til at miste fokus på andre trafikkantgrupper, da de kun er nødsaget til at følge deres egen private bane.
 Projektetes formål, om at undersøge trygheden i området, er blevet undersøgt ved hjælp af sikkerhedsundersøgelser. Dette ligger til grund i, at tryghed ikke kan regnes på. I og med at den undersøgte sikkerhed ikke har været optimal ud fra vores beregninger, og trygheden ikke har været optimal ud fra de lavede interviews, vil der formodes at være en form for sammenhæng. Derfor er projektet mundet ud i et løsningsforslag til et mere sikkert trafikmiljø.
